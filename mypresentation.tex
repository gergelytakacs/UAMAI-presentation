\documentclass{beamer}                              % Use [handout] to disable transitions / Pouzivat opciu [handout] na vypnutie prechodov
\usepackage{textpos} % package for the positioning
\usepackage{eso-pic}
\usepackage{tikz}

\newcommand\AtPagemyLowerRight[1]{\AtPageLowerLeft{%
\put(\LenToUnit{0.0\paperwidth},\LenToUnit{0.0\paperheight}){#1}}}
\AddToShipoutPictureFG{
  \AtPagemyLowerRight{{\includegraphics[height=0.29cm,keepaspectratio]{footer}}}
}%


\mode<presentation>
{
  \usetheme{Boadilla}
  \setbeamercovered{transparent}  % or whatever (possibly just delete it)
}

\usepackage{color}
\definecolor{arduteal}{RGB}{23, 161, 165}
\definecolor{uamai}{RGB}{141, 198, 63}

\makeatletter
\setbeamertemplate{footline}
{
  \leavevmode%
  \hbox{%
  \begin{beamercolorbox}[wd=.333333\paperwidth,ht=2.25ex,dp=1ex,center]{section in head/foot}%
    \usebeamerfont{author in head/foot}\insertshortauthor~~\beamer@ifempty{\insertshortinstitute}{}{\insertshortinstitute}
  \end{beamercolorbox}%
  \begin{beamercolorbox}[wd=.333333\paperwidth,ht=2.25ex,dp=1ex,center]{section in head/foot}%
    \usebeamerfont{title in head/foot}\insertshorttitle
  \end{beamercolorbox}%
  \begin{beamercolorbox}[wd=.333333\paperwidth,ht=2.25ex,dp=1ex,right]{section in head/foot}%
    \usebeamerfont{date in head/foot}\insertshortdate{}\hspace*{2em}
    \insertframenumber{} / \inserttotalframenumber\hspace*{2ex}
  \end{beamercolorbox}}%
  \vskip0pt%
}
\makeatother

\setbeamercolor{toptitlecolor}{fg=black,bg=uamai}
\makeatletter
\setbeamertemplate{frametitle}
{\vskip-3pt
  \leavevmode
  \hbox{%
  \begin{beamercolorbox}[wd=\paperwidth,ht=2.7ex,dp=1.2ex]{toptitlecolor}%
    \raggedright\hspace*{0.4em}\large\insertframetitle
  \end{beamercolorbox}
  }%
}
\makeatother

\addtobeamertemplate{frametitle}{}{%
\begin{tikzpicture}[remember picture,overlay]
\node[anchor=north east,yshift=7.9mm,xshift=4.5mm] at (current page.north east) {\includegraphics[height=2cm]{uamai}};
\end{tikzpicture}}


%
%\addtobeamertemplate{frametitle}{}{%
%


\makeatletter
\definecolor{beamer@blendedblue}{RGB}{141, 198, 63}
\makeatother

 \usepackage{graphicx}
\graphicspath{{./figs/}}
\usepackage{epstopdf} %Converts EPS to PDF so PDFLaTeX can be used
\newcommand{\figone}[0] {1.000\textwidth} %Multiplier for all single full width images
\newcommand{\figtwo}[0] {0.48\textwidth} %Multiplier for all double half width images
\newcommand{\figthree}[0] {0.321\textwidth} %Multiplier for all triple third width images
\newcommand{\figtwoh}[0] {0.3\textheight} %Multiplier for all double half width


\usepackage{subfigure}
\usepackage{amsfonts}
\usepackage{amssymb}
\usepackage{amsmath}
\usepackage{theorem}
\usepackage{epsfig}
\usepackage{mathrsfs} %mathscr
\usepackage[cp1250]{inputenc}
\usepackage[T1]{fontenc}
\usepackage{courier} % To get courier fonts in listings
\usepackage{listings}
\lstset{basicstyle=\normalsize\ttfamily,breaklines=true,numbers=left,xleftmargin=2em,frame=single,framexleftmargin=1.5em}

\lstdefinelanguage{Arduino}[]{C++}      %Defining Arduino
{morekeywords={pgm_read_float,pgm_read_word,PROGMEM,pinMode,digitalWrite,digitalRead,analogReference,analogRead,analogWrite,analogReadResolution,analogWriteResolution,tone,noTone,shiftOut,shiftIn,pulseIn,millis,micros,delay,delayMicroseconds,min,max,abs,constrain,map,pow,sqrt,sin,cos,tan,isAlphaNumeric,isAlpha,isAscii,isWhitespace,isControl,isDigit,isGraph,isLowerCase,isPrintable,isPunct,isSpace,isUpperCase,iisHexadecimalDigit,randomSeed,random,lowByte,highByte,bitRead,bitWrite,
bitSet,bitClear,bit,attachInterrupt,detachInterrupt,interrupts,noInterrupts,Serial,Stream,Keyboard,Mouse},
sensitive=true,
alsoletter={_}
}



\lstdefinelanguage{myArduino}[]{Arduino}      %Defining my own Arduino
{morekeywords={setup,loop,function},
sensitive=true,
alsoletter={_},
numbers=left,
xleftmargin=2em,
frame=single,
framexleftmargin=1.5em
}

%Extended Matlab language
\lstdefinelanguage{myMatlab}[]{Matlab}      {morekeywords={ectrl.exportToC,cost,LTISystem,QuadFunction,MPCController,toExplicit,partition,feedback,fplot,function,arduino,writeDigitalPin,textscan,fminsearch,dlqr,predmodelqp,dlyap,ones,linprog,quadprog,optimset,qpOASES,qpOASES_sequence,sysStruct,probStruct,mpt_control,volume,hull,extreme,mpt_exportc,mpt_getInput,sdpvar,blkdiag,sdpsettings,solvesdp,geomean,double,},
sensitive=true,
alsoletter={_}
numbers=left,
xleftmargin=2em,
frame=single,
framexleftmargin=1.5em
}

\lstdefinelanguage{myPython}[]{Python}
{morekeywords={},
sensitive=true,
alsoletter={_}
numbers=left,
xleftmargin=2em,
frame=single,
framexleftmargin=1.5em}

\lstdefinelanguage{myBash}[]{bash}
{morekeywords={},
sensitive=true,
alsoletter={_}
numbers=left,
xleftmargin=2em,
frame=single,
framexleftmargin=1.5em}

\lstdefinestyle{customc}{
  belowcaptionskip=1\baselineskip,
  breaklines=true,
  xleftmargin=\parindent,
  language=myArduino,
  showstringspaces=false,
  basicstyle=\normalsize\ttfamily,
  keywordstyle=\bfseries\color{green!40!black},
  commentstyle=\itshape\color{purple!40!black},
  identifierstyle=\color{blue},
  stringstyle=\color{orange},
}

\lstdefinestyle{custompython}{
  belowcaptionskip=1\baselineskip,
  breaklines=true,
  xleftmargin=\parindent,
  language=myPython,
  showstringspaces=false,
  basicstyle=\normalsize\ttfamily,
  keywordstyle=\bfseries\color{green!40!black},
  commentstyle=\itshape\color{purple!40!black},
  identifierstyle=\color{blue},
  stringstyle=\color{orange},
}

\lstdefinestyle{custommatlab}{
  belowcaptionskip=1\baselineskip,
  breaklines=true,
  xleftmargin=\parindent,
  language=myMatlab,
  showstringspaces=false,
  basicstyle=\normalsize\ttfamily,
  keywordstyle=\bfseries\color{green!40!black},
  commentstyle=\itshape\color{purple!40!black},
  identifierstyle=\color{blue},
  stringstyle=\color{orange},
 }


\lstnewenvironment{arduino}[1]{
\lstset{
language=myArduino,
caption=#1,
label=#1,
style=customc}
}
{
}

%Plain with no numbering
\lstnewenvironment{ardu}{
\lstset{
language=myArduino,
style=customc}
}
{
}

\lstnewenvironment{matlab}[1]{
\lstset{
language=myMatlab,
caption=#1,
label=#1,
style=custommatlab}
}
{
}

\lstnewenvironment{mtlb}{
\lstset{
language=myMatlab,
style=custommatlab}
}
{
}

%----------------------------------------------------
\newcommand{\muaompc}{$\mu${AO-MPC}}

\renewcommand{\vec}[1]{\boldsymbol{\mathrm{#1}}} %Bold vectors instead of arrows

%This custom command defines how the literal menus look like.
\newcommand{\gui}[1]{{\emph{#1}}} %Gui commands, icon names, buttons

% All code, functions, variables are typed like this
\newcommand{\code}[1]{{\lstinline[columns=fixed]{#1}}} %Shorthand for code


\usepackage[framed,numbered,autolinebreaks,useliterate]{mcode}
                                 % Contains settings for IAMAI / Obsahuje nastavenia pre UAMAI

%\dualscreen                                        % For showing notes on the second screen / Pre poznamky na druhej obrazovke

\title[Event or Presentation title]                 % Goes into footer only
{My awesome presentation}                           % Main title
\subtitle{A 15 minute LaTeX Beamer extravaganza}    % (optional, use only with long paper titles)
\author[]{Bc. Jo\v{z}ko Mrkvi\v{c}ka}               % Name / Meno
\date[06.06.2019]{}                                 % Date / Dátum
\slideheader{EN}                                    % Logo in the header / Logo v hlavicke: SK, EN
\slidefooter{EN}{STU}                               % Institucia a jazykova varianta / Institution and language version: SK, EN and STU, SjF



\begin{document}

\titleslide[prof. Voldemort, PhD.]{EN}{STU}    % Title slide / Titulkovy slide. Language / Jazyk: SK,EN; Logo STU, SjF. First optional argument [] is the supervisor / Prvý argument je vedúci

%-----------------------------------
%*** CONTENT BEGIN / ZACAT OBSAH ***
%-----------------------------------
\begin{frame}{Classic itemized list}
This is some normal text with some items:
  \begin{itemize}
    \item List item one,
    \item List item two,
    \item and list item three.
  \end{itemize}
  And some more text.

  \note{This is a slide note}

\end{frame}

\begin{frame}{Blocks}
You can also highlight information in a so-called block
\begin{block}{This is a block}
Some information worth remembering.
\end{block}
\end{frame}



\begin{frame}[fragile]{Code and fragile frames}
Beamer does not like code listings very much, so you will need to use the {\bf fragile} modifier. This is an example of some Arduino code:
\begin{ardu}
void setup() {
     // Pin D13 out
}

void loop() {
     // turn LED on
     // wait 1 s
     // turn LED off
     // wait 1 s
}
\end{ardu}
\end{frame}


\begin{frame}{Columns}
Sometimes it is a good idea to divide the frame into two columns. You can use other environments in frames:
\begin{columns}[T] % align columns
\begin{column}{.48\textwidth}
This is the content in the first (left) column.

Yet more information.

Chicken, chicken, chicken.
\end{column}
\begin{column}{.48\textwidth}
This is the content in the second (right) column.

Yet more information.


Chicken, chicken, chicken.
\end{column}%
\end{columns}
\end{frame}

\part{{\bf Parts} \protect\\ This is a stand-alone page separating parts. Use it sparingly, only for long presentations or lectures (>30-45 min).}
\frame{\partpage}

\begin{frame}{Figures}
Figures work the same exact way as in any other \LaTeX document:
\begin{figure}
\centering
  \includegraphics[width=50mm]{borat}\\
\end{figure}
This of course includes the subfigure environment.
\end{frame}

\begin{frame}{References}
You can and should use references. References are cited like in any other \LaTeX document, so we have \cite{Takacs2016b} or \cite{Asato2015,Stark2013} etc. References are inserted at the end of the presentation, are automatically broken up to frames and are not numbered.
\end{frame}

\begin{frame}{Pauses}
Pauses serve to gradually reveal parts of your frame. This will create more pages in the PDF (without advancing the frame counter) thereby achieving the desired result.
\pause
\begin{itemize}
\item Then you can go on the next portion of the information...
\pause
\item ...and the next one
\end{itemize}
\end{frame}

\begin{frame}{Overlays}
Overlays essentially add more control to pause-like behavior when you want to reveal specific parts of your frame. By using overlay commands such as \code{<1-2,4,5->}, you can tell exactly when the part shall appear. The numbers in the brackets specify when the item should appear, the hyphen before tells \LaTeX that it should be there from the beginning, the hyphen after means that it should stay up to the end.

Here is an example. \onslide<1->{You can} \onslide<2->{reveal this} \onslide<3->{text slowly} or use the overlays in any other environment:

\begin{itemize}
\item<1-> Foo
\item<3-> Bar
\item<-2> Baz
\item<1,4> Qux
\end{itemize}
\end{frame}

\begin{frame}[allowframebreaks]{Framebreaks}

You can create content and then break it up to several sub-slides by the \code{allowframebreaks} modifier. If you want to manually trigger a break, you then use \code{framebreak}. If you wish, you can just continue on your text and let \LaTeX decide when to break your frame to more slides (e.g. pages in the PDF).

\framebreak

\LaTeX will hold page numbering on the given slide, keep the title and append a roman numeral, as to indicate that the content continues. (It is a nasty little trick to decrease complete frame numbering, but still have the slides you need.)

\end{frame}

\begin{frame}{Highlights}
\begin{itemize}
\item Use the \code{important} command to create text that is \important{important}. Alternatively you can use the \code{alert} command that is valid in Beamer to \alert{achieve the same results}. The original color for alerts in beamer is red, which would look very ugly here.
\item Use the \code{veryimportant} command to create text that is \veryimportant{very important}.
\end{itemize}
\end{frame}

\begin{frame}{Notes on the second screen}
\begin{itemize}
\item In many situations it is good to see your own notes for the speech on a second screen. This screen is facing towards the presenter, away from the audience and contains material separate to the main presentation. To use this uncomment the \code{dualscreen} command in the preamble.

\item You must then consult \url{https://www.scivision.dev/beamer-latex-dual-display-pdf-notes/}, where you can find information on how to set up your computer for various operating systems to display notes. The binary executable installers of \code{pympress} for Windows are found here: \url{https://github.com/Cimbali/pympress/releases/tag/v1.2.0}. You can swap screens by pressing the \code{s} button.

\item Notes can be entered using the \code{note} command.

\end{itemize}

\note{This is a note that will be displayed on the second screen}

\end{frame}

\begin{frame}[fragile]{Be creative!}
Combine the above tricks to create visually engaging presentations!
\begin{columns}[T] % align columns
\begin{column}{.48\textwidth}
\begin{ardu}
void setup() {
     // Pin D13 out
}

void loop() {
     // turn LED on
     // wait 1 s
     // turn LED off
     // wait 1 s
}
\end{ardu}
\end{column}
\begin{column}{.48\textwidth}
\begin{block}{Important information}
Very important information
\end{block}
\begin{itemize}
\item List item one
\item List item two
\item List item three
\end{itemize}
\begin{figure}
\centering
  \includegraphics[width=20mm]{borat}\\
\end{figure}
\end{column}%
\end{columns}
\end{frame}




\begin{frame}[allowframebreaks]{Prezentácia záverečnej práce (BP/DP)}
\begin{itemize}
  \item Dĺžka prezentácie je 15 min ani o minútu dlhšie.
  \item Počet strán cca. 1-2 na minútu, odporúčané je teda 10-20.
  \item Premietaná prezentácia slúži pre audienciu nie pre Vás. Spolu s tým čo hovoríte tvorí celok.
  \item Neprepĺňajte slide informáciami. (Ako napríklad tu.) Je absolútne dovolené mať jednu vetu resp. jednu fotku na slide a potom slovne vysvetliť všetko ostatné.
 \item Úvodná strana obsahuje názov práce a Vaše meno
 \item Žiadne úvody o všeobecnej teórii
 \item Musíte vysvetliť motiváciu. Motiváciu na napísanie BP/DP a motiváciu pre komisiu prečo dávať pozor
  \item Prezentujte hlavne svoj prínos, vlastnú prácu a vlastné výsledky
  \item Z prezentácie musí byť jasné čo ste robili vy, čo už ste mali dané, atď.
  \item Fotografie reálnych výstupov sú vysoko vítané...
  \item Začínajte s tím prečo ste robili to čo robíte, potom aké boli východiská, ďalej ako ste samotný problém riešili a na záver zhodnoťte dosiahnuté ciele, prípadne dajte ešte krátku víziu čo by sa dalo robiť do budúcnosti (ak tam je potenciál - samozrejme).
  \item Posledná snímka (slide) Poďakovanie za pozornosť...
 \item Po poďakovaní čakajte na vyzvanie a potom bude prezentácia pokračovať vopred pripravenými odpoveďami na otázky oponenta. Štýlom Otázka oponenta a pod ňou Vaša odpoveď... Môžete vložiť pár slideov až po poďakovaní kde uvediete otázky oponenta a Vaše odpovede na nich
 \item Ak máte videá z prezentácie dajte ich na záver svojej prezentácie, môže byť aj ako externý zdroj.

\end{itemize}
\end{frame}


\begin{frame}{Prezentácia záverečnej práce (Minimová páca)}

Okrem všeobecne platných princípov pre prezentácie BP/DP na minimovú prácu platí aj

\begin{itemize}
 \item Uvediete podstatný obsah svojej práce a jej výsledky
 \item v rozsahu najviac 25 min, skôr 20 min.
 \item Po závere a ešte pred poďakovaním vložíte tézy svojej práce.
 \item Tézy sú testovateľné vedecké hypotézy nie ``zadania``!
\end{itemize}
\end{frame}

\begin{frame}{Prezentácia záverečnej práce (Dizertačná páca)}

Okrem všeobecne platných princípov pre prezentácie BP/DP dizertačnú prácu platí aj

\begin{itemize}
 \item Uvediete podstatný obsah svojej práce a hlavne jej výsledky
 \item splnenie cieľov zadania
 \item prínos práce
 \item používajte odvolávky na vlastnú prácu
 \item v rozsahu najviac 25 min, skôr 20 min.
\end{itemize}
\end{frame}


%-----------------------------------
%*** CONTENT END / KONIEC OBSAHU ***
%-----------------------------------

\thankyouslide{jmrkvicka@stuba.sk}{EN} % Language / jazyk: SK, EN; Your e-mail or www / Tvoja e-mail alebo www adresa

%-----------------------------------
%*** OPTIONAL / NEPOVINNE ***
%-----------------------------------

% RESPOND TO QUESTIONS / ODPOVEDAT NA OTAZKY
\begin{reviewerquestions}{EN} % Use this to answer reviewer / Používať na pripravené odpovede na oponentské otázky EN, SK

{\it How did you calculate the mass of the sun?}

The sun's mass has been found using Newton's law of gravitation.

\framebreak

{\it Why is the sky blue?}

Blue light is scattered in all directions by the tiny molecules of air in Earth's atmosphere. Blue is scattered more than other colors because it travels as shorter, smaller waves. This is why we see a blue sky most of the time.

\framebreak

{\it How many five year olds can you realistically take in a fight?}

About 31.

\end{reviewerquestions}

% REFERENCES / BIBLIOGRAFIA
\referencesslide{Bibliography}{EN}          % Language / jazyk: SK, EN; Name of *.bib file / Nazov *.bib suboru

\end{document}





