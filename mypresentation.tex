\documentclass{beamer}
\usepackage{textpos} % package for the positioning
\usepackage{eso-pic}
\usepackage{tikz}

\newcommand\AtPagemyLowerRight[1]{\AtPageLowerLeft{%
\put(\LenToUnit{0.0\paperwidth},\LenToUnit{0.0\paperheight}){#1}}}
\AddToShipoutPictureFG{
  \AtPagemyLowerRight{{\includegraphics[height=0.29cm,keepaspectratio]{footer}}}
}%


\mode<presentation>
{
  \usetheme{Boadilla}
  \setbeamercovered{transparent}  % or whatever (possibly just delete it)
}

\usepackage{color}
\definecolor{arduteal}{RGB}{23, 161, 165}
\definecolor{uamai}{RGB}{141, 198, 63}

\makeatletter
\setbeamertemplate{footline}
{
  \leavevmode%
  \hbox{%
  \begin{beamercolorbox}[wd=.333333\paperwidth,ht=2.25ex,dp=1ex,center]{section in head/foot}%
    \usebeamerfont{author in head/foot}\insertshortauthor~~\beamer@ifempty{\insertshortinstitute}{}{\insertshortinstitute}
  \end{beamercolorbox}%
  \begin{beamercolorbox}[wd=.333333\paperwidth,ht=2.25ex,dp=1ex,center]{section in head/foot}%
    \usebeamerfont{title in head/foot}\insertshorttitle
  \end{beamercolorbox}%
  \begin{beamercolorbox}[wd=.333333\paperwidth,ht=2.25ex,dp=1ex,right]{section in head/foot}%
    \usebeamerfont{date in head/foot}\insertshortdate{}\hspace*{2em}
    \insertframenumber{} / \inserttotalframenumber\hspace*{2ex}
  \end{beamercolorbox}}%
  \vskip0pt%
}
\makeatother

\setbeamercolor{toptitlecolor}{fg=black,bg=uamai}
\makeatletter
\setbeamertemplate{frametitle}
{\vskip-3pt
  \leavevmode
  \hbox{%
  \begin{beamercolorbox}[wd=\paperwidth,ht=2.7ex,dp=1.2ex]{toptitlecolor}%
    \raggedright\hspace*{0.4em}\large\insertframetitle
  \end{beamercolorbox}
  }%
}
\makeatother

\addtobeamertemplate{frametitle}{}{%
\begin{tikzpicture}[remember picture,overlay]
\node[anchor=north east,yshift=7.9mm,xshift=4.5mm] at (current page.north east) {\includegraphics[height=2cm]{uamai}};
\end{tikzpicture}}


%
%\addtobeamertemplate{frametitle}{}{%
%


\makeatletter
\definecolor{beamer@blendedblue}{RGB}{141, 198, 63}
\makeatother

 \usepackage{graphicx}
\graphicspath{{./figs/}}
\usepackage{epstopdf} %Converts EPS to PDF so PDFLaTeX can be used
\newcommand{\figone}[0] {1.000\textwidth} %Multiplier for all single full width images
\newcommand{\figtwo}[0] {0.48\textwidth} %Multiplier for all double half width images
\newcommand{\figthree}[0] {0.321\textwidth} %Multiplier for all triple third width images
\newcommand{\figtwoh}[0] {0.3\textheight} %Multiplier for all double half width


\usepackage{subfigure}
\usepackage{amsfonts}
\usepackage{amssymb}
\usepackage{amsmath}
\usepackage{theorem}
\usepackage{epsfig}
\usepackage{mathrsfs} %mathscr
\usepackage[cp1250]{inputenc}
\usepackage[T1]{fontenc}
\usepackage{courier} % To get courier fonts in listings
\usepackage{listings}
\lstset{basicstyle=\normalsize\ttfamily,breaklines=true,numbers=left,xleftmargin=2em,frame=single,framexleftmargin=1.5em}

\lstdefinelanguage{Arduino}[]{C++}      %Defining Arduino
{morekeywords={pgm_read_float,pgm_read_word,PROGMEM,pinMode,digitalWrite,digitalRead,analogReference,analogRead,analogWrite,analogReadResolution,analogWriteResolution,tone,noTone,shiftOut,shiftIn,pulseIn,millis,micros,delay,delayMicroseconds,min,max,abs,constrain,map,pow,sqrt,sin,cos,tan,isAlphaNumeric,isAlpha,isAscii,isWhitespace,isControl,isDigit,isGraph,isLowerCase,isPrintable,isPunct,isSpace,isUpperCase,iisHexadecimalDigit,randomSeed,random,lowByte,highByte,bitRead,bitWrite,
bitSet,bitClear,bit,attachInterrupt,detachInterrupt,interrupts,noInterrupts,Serial,Stream,Keyboard,Mouse},
sensitive=true,
alsoletter={_}
}



\lstdefinelanguage{myArduino}[]{Arduino}      %Defining my own Arduino
{morekeywords={setup,loop,function},
sensitive=true,
alsoletter={_},
numbers=left,
xleftmargin=2em,
frame=single,
framexleftmargin=1.5em
}

%Extended Matlab language
\lstdefinelanguage{myMatlab}[]{Matlab}      {morekeywords={ectrl.exportToC,cost,LTISystem,QuadFunction,MPCController,toExplicit,partition,feedback,fplot,function,arduino,writeDigitalPin,textscan,fminsearch,dlqr,predmodelqp,dlyap,ones,linprog,quadprog,optimset,qpOASES,qpOASES_sequence,sysStruct,probStruct,mpt_control,volume,hull,extreme,mpt_exportc,mpt_getInput,sdpvar,blkdiag,sdpsettings,solvesdp,geomean,double,},
sensitive=true,
alsoletter={_}
numbers=left,
xleftmargin=2em,
frame=single,
framexleftmargin=1.5em
}

\lstdefinelanguage{myPython}[]{Python}
{morekeywords={},
sensitive=true,
alsoletter={_}
numbers=left,
xleftmargin=2em,
frame=single,
framexleftmargin=1.5em}

\lstdefinelanguage{myBash}[]{bash}
{morekeywords={},
sensitive=true,
alsoletter={_}
numbers=left,
xleftmargin=2em,
frame=single,
framexleftmargin=1.5em}

\lstdefinestyle{customc}{
  belowcaptionskip=1\baselineskip,
  breaklines=true,
  xleftmargin=\parindent,
  language=myArduino,
  showstringspaces=false,
  basicstyle=\normalsize\ttfamily,
  keywordstyle=\bfseries\color{green!40!black},
  commentstyle=\itshape\color{purple!40!black},
  identifierstyle=\color{blue},
  stringstyle=\color{orange},
}

\lstdefinestyle{custompython}{
  belowcaptionskip=1\baselineskip,
  breaklines=true,
  xleftmargin=\parindent,
  language=myPython,
  showstringspaces=false,
  basicstyle=\normalsize\ttfamily,
  keywordstyle=\bfseries\color{green!40!black},
  commentstyle=\itshape\color{purple!40!black},
  identifierstyle=\color{blue},
  stringstyle=\color{orange},
}

\lstdefinestyle{custommatlab}{
  belowcaptionskip=1\baselineskip,
  breaklines=true,
  xleftmargin=\parindent,
  language=myMatlab,
  showstringspaces=false,
  basicstyle=\normalsize\ttfamily,
  keywordstyle=\bfseries\color{green!40!black},
  commentstyle=\itshape\color{purple!40!black},
  identifierstyle=\color{blue},
  stringstyle=\color{orange},
 }


\lstnewenvironment{arduino}[1]{
\lstset{
language=myArduino,
caption=#1,
label=#1,
style=customc}
}
{
}

%Plain with no numbering
\lstnewenvironment{ardu}{
\lstset{
language=myArduino,
style=customc}
}
{
}

\lstnewenvironment{matlab}[1]{
\lstset{
language=myMatlab,
caption=#1,
label=#1,
style=custommatlab}
}
{
}

\lstnewenvironment{mtlb}{
\lstset{
language=myMatlab,
style=custommatlab}
}
{
}

%----------------------------------------------------
\newcommand{\muaompc}{$\mu${AO-MPC}}

\renewcommand{\vec}[1]{\boldsymbol{\mathrm{#1}}} %Bold vectors instead of arrows

%This custom command defines how the literal menus look like.
\newcommand{\gui}[1]{{\emph{#1}}} %Gui commands, icon names, buttons

% All code, functions, variables are typed like this
\newcommand{\code}[1]{{\lstinline[columns=fixed]{#1}}} %Shorthand for code


\usepackage[framed,numbered,autolinebreaks,useliterate]{mcode}



\title[Event or Presentation title] % (optional, use only with long paper titles)
{My awesome presentation:\\ a LaTeX Beamer extravaganza}
%\subtitle{}


\author[] % (optional, use only with lots of authors)
{prof. Ing. Jo\v{z}ko Mrkvi\v{c}ka, PhD.}
%{F.~Author\inst{1} \and S.~Another\inst{2}}
% - Give the names in the same order as the appear in the paper.
% - Use the \inst{?} command only if the authors have different
%   affiliation.

\date[13.04.2018] % (optional, should be abbreviation of conference name)
{}
% - Either use conference name or its abbreviation.
% - Not really informative to the audience, more for people (including
%   yourself) who are reading the slides online

\titlegraphic{\includegraphics[height=3cm]{stu}\hspace*{1cm}~%
   \includegraphics[height=3cm]{uamai}
}

\subject{}
% This is only inserted into the PDF information catalog. Can be left
% out.

% Delete this, if you do not want the table of contents to pop up at
% the beginning of each subsection:

\AtBeginSection[]
{
  \begin{frame}
  \frametitle{Contents}
  \tiny{\tableofcontents[currentsection]}
  \end{frame}
}

% If you wish to uncover everything in a step-wise fashion, uncomment
% the following command:
%\beamerdefaultoverlayspecification{<+->}

\begin{document}
\lstset{aboveskip=3pt, belowskip=3pt} %Space above and below lstlist boxes
%\nologo
\begin{frame}%[plain]
  \titlepage
\end{frame}

%\begin{frame}[plain]{Outline}
%\tableofcontents[part=1]
%\end{frame}
%\begin{frame}
%\frametitle{Outline}
%\tableofcontents[part=1,pausesections]
%\end{frame}

\begin{frame}{Classic itemized list}
This is some normal text with some items:
  \begin{itemize}
    \item List item one,
    \item List item two,
    \item and list item three.
  \end{itemize}
  And some more text.
\end{frame}

\begin{frame}{Blocks}
You can also highlight information
\begin{block}{This is a block}
Some information worth remembering.
\end{block}
\end{frame}



\begin{frame}[fragile]{Code and fragile frames}
Beamer does not like code listings very much, so you will need to use the {\bf fragile} modifier. This is an example of some Arduino code:
\begin{ardu}
void setup() {
     // Pin D13 out
}

void loop() {
     // turn LED on
     // wait 1 s
     // turn LED off
     // wait 1 s
}
\end{ardu}
\end{frame}


\begin{frame}{Columns}
Sometimes it is a good idea to divide the frame into two columns. You can use other environments in frames:
\begin{columns}[T] % align columns
\begin{column}{.48\textwidth}
This is the content in the first (left) column.

Yet more information.

Chicken, chicken, chicken.
\end{column}
\begin{column}{.48\textwidth}
This is the content in the second (right) column.

Yet more information.


Chicken, chicken, chicken.
\end{column}%
\end{columns}
\end{frame}

\part{{\bf Parts} \protect\\ This is a stand-alone page separating parts. Use it sparingly, only for long presentations or lectures (>45 min).}
\frame{\partpage}

\begin{frame}{Figures}
Figures work the same exact way as in any other \LaTeX document:
\begin{figure}
\centering
  \includegraphics[width=50mm]{borat}\\
\end{figure}
This of course includes the subfigure environment.
\end{frame}




\begin{frame}[fragile]{Be creative!}
Combine the above tricks to create visually engaging presentations!
\begin{columns}[T] % align columns
\begin{column}{.48\textwidth}
\begin{ardu}
void setup() {
     // Pin D13 out
}

void loop() {
     // turn LED on
     // wait 1 s
     // turn LED off
     // wait 1 s
}
\end{ardu}
\end{column}
\begin{column}{.48\textwidth}
\begin{block}{Important information}
Very important information
\end{block}
\begin{itemize}
\item List item one
\item List item two
\item List item three
\end{itemize}
\begin{figure}
\centering
  \includegraphics[width=20mm]{borat}\\
\end{figure}
\end{column}%
\end{columns}
\end{frame}

\begin{frame}{Prezentácia záverečnej práce (BP/DP) (1/2)}
\begin{itemize}
  \item Počet strán 10-10, odporúčané 15.
  \item Strany musia byť očíslované
 \item Úvodná strana obsahuje, logo fakulty a ústavu, názov práce, Vaše meno a Meno školiteľa plus dátum prezentácie.
 \item Nasledujúca strana, krátky abstrakt práce (max. 5 viet) a stručný popis čo prezentácia bude prezentovať (cez odrážky - bodovo)
 \item V práci prezentujte HLAVNE svoj prínos žiadne úvody o všeobecnej teórii
  \item fotografie reálnych výstupov sú vysoko vítané...

\end{itemize}

\end{frame}


\begin{frame}{Prezentácia záverečnej práce (BP/DP) (1/2)}
\begin{itemize}
 \item Začínajte s tím prečo ste robili to čo robíte, potom aké boli východiská, ďalej ako ste samotný problém riešili a na záver zhodnoťte dosiahnuté ciele, prípadne dajte ešte krátku víziu čo by sa dalo robiť do budúcnosti (ak tam je potenciál - samozrejme).
 \item Posledná snímka (slide) Poďakovanie za pozornosť...
 \item Po poďakovaní čakajte na vyzvanie a potom bude prezentácia pokračovať vopred pripravenými odpoveďami na otázky oponenta. Štýlom Otázka oponenta a pod ňou Vaša odpoveď...
 \item Ak máte videá z prezentácie dajte ich na záver svojej prezentácie, môže byť aj ako externý zdroj.
\end{itemize}

\end{frame}


\begin{frame}
\vspace{8em}
{\Huge Thank you for your attention!}

\vspace{8em}
Please feel free to contact me at
\url{jozko.mrkvicka@stuba.sk}
\end{frame}




\end{document}





