\usepackage{textpos} % package for the positioning
\usepackage{eso-pic}
\usepackage{tikz}

\newcommand\AtPagemyLowerRight[1]{\AtPageLowerLeft{%
\put(\LenToUnit{0.0\paperwidth},\LenToUnit{0.0\paperheight}){#1}}}
\AddToShipoutPictureFG{
  \AtPagemyLowerRight{{\includegraphics[height=0.29cm,keepaspectratio]{footer}}}
}%


\mode<presentation>
{
  \usetheme{Boadilla}
  \setbeamercovered{transparent}  % or whatever (possibly just delete it)
}

\usepackage{color}
\definecolor{arduteal}{RGB}{23, 161, 165}
\definecolor{uamai}{RGB}{141, 198, 63}

\makeatletter
\setbeamertemplate{footline}
{
  \leavevmode%
  \hbox{%
  \begin{beamercolorbox}[wd=.333333\paperwidth,ht=2.25ex,dp=1ex,center]{section in head/foot}%
    \usebeamerfont{author in head/foot}\insertshortauthor~~\beamer@ifempty{\insertshortinstitute}{}{\insertshortinstitute}
  \end{beamercolorbox}%
  \begin{beamercolorbox}[wd=.333333\paperwidth,ht=2.25ex,dp=1ex,center]{section in head/foot}%
    \usebeamerfont{title in head/foot}\insertshorttitle
  \end{beamercolorbox}%
  \begin{beamercolorbox}[wd=.333333\paperwidth,ht=2.25ex,dp=1ex,right]{section in head/foot}%
    \usebeamerfont{date in head/foot}\insertshortdate{}\hspace*{2em}
    \insertframenumber{} / \inserttotalframenumber\hspace*{2ex}
  \end{beamercolorbox}}%
  \vskip0pt%
}
\makeatother

\setbeamercolor{toptitlecolor}{fg=black,bg=uamai}
\makeatletter
\setbeamertemplate{frametitle}
{\vskip-3pt
  \leavevmode
  \hbox{%
  \begin{beamercolorbox}[wd=\paperwidth,ht=2.7ex,dp=1.2ex]{toptitlecolor}%
    \raggedright\hspace*{0.4em}\large\insertframetitle
  \end{beamercolorbox}
  }%
}
\makeatother

\addtobeamertemplate{frametitle}{}{%
\begin{tikzpicture}[remember picture,overlay]
\node[anchor=north east,yshift=7.9mm,xshift=4.5mm] at (current page.north east) {\includegraphics[height=2cm]{uamai}};
\end{tikzpicture}}


%
%\addtobeamertemplate{frametitle}{}{%
%


\makeatletter
\definecolor{beamer@blendedblue}{RGB}{141, 198, 63}
\makeatother

 \usepackage{graphicx}
\graphicspath{{./figs/}}
\usepackage{epstopdf} %Converts EPS to PDF so PDFLaTeX can be used
\newcommand{\figone}[0] {1.000\textwidth} %Multiplier for all single full width images
\newcommand{\figtwo}[0] {0.48\textwidth} %Multiplier for all double half width images
\newcommand{\figthree}[0] {0.321\textwidth} %Multiplier for all triple third width images
\newcommand{\figtwoh}[0] {0.3\textheight} %Multiplier for all double half width


\usepackage{subfigure}
\usepackage{amsfonts}
\usepackage{amssymb}
\usepackage{amsmath}
\usepackage{theorem}
\usepackage{epsfig}
\usepackage{mathrsfs} %mathscr
\usepackage[cp1250]{inputenc}
\usepackage[T1]{fontenc}
\usepackage{courier} % To get courier fonts in listings
\usepackage{listings}
\lstset{basicstyle=\normalsize\ttfamily,breaklines=true,numbers=left,xleftmargin=2em,frame=single,framexleftmargin=1.5em}

\lstdefinelanguage{Arduino}[]{C++}      %Defining Arduino
{morekeywords={pgm_read_float,pgm_read_word,PROGMEM,pinMode,digitalWrite,digitalRead,analogReference,analogRead,analogWrite,analogReadResolution,analogWriteResolution,tone,noTone,shiftOut,shiftIn,pulseIn,millis,micros,delay,delayMicroseconds,min,max,abs,constrain,map,pow,sqrt,sin,cos,tan,isAlphaNumeric,isAlpha,isAscii,isWhitespace,isControl,isDigit,isGraph,isLowerCase,isPrintable,isPunct,isSpace,isUpperCase,iisHexadecimalDigit,randomSeed,random,lowByte,highByte,bitRead,bitWrite,
bitSet,bitClear,bit,attachInterrupt,detachInterrupt,interrupts,noInterrupts,Serial,Stream,Keyboard,Mouse},
sensitive=true,
alsoletter={_}
}



\lstdefinelanguage{myArduino}[]{Arduino}      %Defining my own Arduino
{morekeywords={setup,loop,function},
sensitive=true,
alsoletter={_},
numbers=left,
xleftmargin=2em,
frame=single,
framexleftmargin=1.5em
}

%Extended Matlab language
\lstdefinelanguage{myMatlab}[]{Matlab}      {morekeywords={ectrl.exportToC,cost,LTISystem,QuadFunction,MPCController,toExplicit,partition,feedback,fplot,function,arduino,writeDigitalPin,textscan,fminsearch,dlqr,predmodelqp,dlyap,ones,linprog,quadprog,optimset,qpOASES,qpOASES_sequence,sysStruct,probStruct,mpt_control,volume,hull,extreme,mpt_exportc,mpt_getInput,sdpvar,blkdiag,sdpsettings,solvesdp,geomean,double,},
sensitive=true,
alsoletter={_}
numbers=left,
xleftmargin=2em,
frame=single,
framexleftmargin=1.5em
}

\lstdefinelanguage{myPython}[]{Python}
{morekeywords={},
sensitive=true,
alsoletter={_}
numbers=left,
xleftmargin=2em,
frame=single,
framexleftmargin=1.5em}

\lstdefinelanguage{myBash}[]{bash}
{morekeywords={},
sensitive=true,
alsoletter={_}
numbers=left,
xleftmargin=2em,
frame=single,
framexleftmargin=1.5em}

\lstdefinestyle{customc}{
  belowcaptionskip=1\baselineskip,
  breaklines=true,
  xleftmargin=\parindent,
  language=myArduino,
  showstringspaces=false,
  basicstyle=\normalsize\ttfamily,
  keywordstyle=\bfseries\color{green!40!black},
  commentstyle=\itshape\color{purple!40!black},
  identifierstyle=\color{blue},
  stringstyle=\color{orange},
}

\lstdefinestyle{custompython}{
  belowcaptionskip=1\baselineskip,
  breaklines=true,
  xleftmargin=\parindent,
  language=myPython,
  showstringspaces=false,
  basicstyle=\normalsize\ttfamily,
  keywordstyle=\bfseries\color{green!40!black},
  commentstyle=\itshape\color{purple!40!black},
  identifierstyle=\color{blue},
  stringstyle=\color{orange},
}

\lstdefinestyle{custommatlab}{
  belowcaptionskip=1\baselineskip,
  breaklines=true,
  xleftmargin=\parindent,
  language=myMatlab,
  showstringspaces=false,
  basicstyle=\normalsize\ttfamily,
  keywordstyle=\bfseries\color{green!40!black},
  commentstyle=\itshape\color{purple!40!black},
  identifierstyle=\color{blue},
  stringstyle=\color{orange},
 }


\lstnewenvironment{arduino}[1]{
\lstset{
language=myArduino,
caption=#1,
label=#1,
style=customc}
}
{
}

%Plain with no numbering
\lstnewenvironment{ardu}{
\lstset{
language=myArduino,
style=customc}
}
{
}

\lstnewenvironment{matlab}[1]{
\lstset{
language=myMatlab,
caption=#1,
label=#1,
style=custommatlab}
}
{
}

\lstnewenvironment{mtlb}{
\lstset{
language=myMatlab,
style=custommatlab}
}
{
}

%----------------------------------------------------
\newcommand{\muaompc}{$\mu${AO-MPC}}

\renewcommand{\vec}[1]{\boldsymbol{\mathrm{#1}}} %Bold vectors instead of arrows

%This custom command defines how the literal menus look like.
\newcommand{\gui}[1]{{\emph{#1}}} %Gui commands, icon names, buttons

% All code, functions, variables are typed like this
\newcommand{\code}[1]{{\lstinline[columns=fixed]{#1}}} %Shorthand for code


\usepackage[framed,numbered,autolinebreaks,useliterate]{mcode}
